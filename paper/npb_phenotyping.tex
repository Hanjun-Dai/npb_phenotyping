
%% bare_conf.tex
%% V1.3
%% 2007/01/11
%% by Michael Shell
%% See:
%% http://www.michaelshell.org/
%% for current contact information.
%%
%% This is a skeleton file demonstrating the use of IEEEtran.cls
%% (requires IEEEtran.cls version 1.7 or later) with an IEEE conference paper.
%%
%% Support sites:
%% http://www.michaelshell.org/tex/ieeetran/
%% http://www.ctan.org/tex-archive/macros/latex/contrib/IEEEtran/
%% and
%% http://www.ieee.org/

%%*************************************************************************
%% Legal Notice:
%% This code is offered as-is without any warranty either expressed or
%% implied; without even the implied warranty of MERCHANTABILITY or
%% FITNESS FOR A PARTICULAR PURPOSE! 
%% User assumes all risk.
%% In no event shall IEEE or any contributor to this code be liable for
%% any damages or losses, including, but not limited to, incidental,
%% consequential, or any other damages, resulting from the use or misuse
%% of any information contained here.
%%
%% All comments are the opinions of their respective authors and are not
%% necessarily endorsed by the IEEE.
%%
%% This work is distributed under the LaTeX Project Public License (LPPL)
%% ( http://www.latex-project.org/ ) version 1.3, and may be freely used,
%% distributed and modified. A copy of the LPPL, version 1.3, is included
%% in the base LaTeX documentation of all distributions of LaTeX released
%% 2003/12/01 or later.
%% Retain all contribution notices and credits.
%% ** Modified files should be clearly indicated as such, including  **
%% ** renaming them and changing author support contact information. **
%%
%% File list of work: IEEEtran.cls, IEEEtran_HOWTO.pdf, bare_adv.tex,
%%                    bare_conf.tex, bare_jrnl.tex, bare_jrnl_compsoc.tex
%%*************************************************************************

% *** Authors should verify (and, if needed, correct) their LaTeX system  ***
% *** with the testflow diagnostic prior to trusting their LaTeX platform ***
% *** with production work. IEEE's font choices can trigger bugs that do  ***
% *** not appear when using other class files.                            ***
% The testflow support page is at:
% http://www.michaelshell.org/tex/testflow/



% Note that the a4paper option is mainly intended so that authors in
% countries using A4 can easily print to A4 and see how their papers will
% look in print - the typesetting of the document will not typically be
% affected with changes in paper size (but the bottom and side margins will).
% Use the testflow package mentioned above to verify correct handling of
% both paper sizes by the user's LaTeX system.
%
% Also note that the "draftcls" or "draftclsnofoot", not "draft", option
% should be used if it is desired that the figures are to be displayed in
% draft mode.
%
\documentclass[conference]{IEEEtran}
% Add the compsoc option for Computer Society conferences.
%
% If IEEEtran.cls has not been installed into the LaTeX system files,
% manually specify the path to it like:
% \documentclass[conference]{../sty/IEEEtran}





% Some very useful LaTeX packages include:
% (uncomment the ones you want to load)


% *** MISC UTILITY PACKAGES ***
%
%\usepackage{ifpdf}
% Heiko Oberdiek's ifpdf.sty is very useful if you need conditional
% compilation based on whether the output is pdf or dvi.
% usage:
% \ifpdf
%   % pdf code
% \else
%   % dvi code
% \fi
% The latest version of ifpdf.sty can be obtained from:
% http://www.ctan.org/tex-archive/macros/latex/contrib/oberdiek/
% Also, note that IEEEtran.cls V1.7 and later provides a builtin
% \ifCLASSINFOpdf conditional that works the same way.
% When switching from latex to pdflatex and vice-versa, the compiler may
% have to be run twice to clear warning/error messages.






% *** CITATION PACKAGES ***
%
%\usepackage{cite}
% cite.sty was written by Donald Arseneau
% V1.6 and later of IEEEtran pre-defines the format of the cite.sty package
% \cite{} output to follow that of IEEE. Loading the cite package will
% result in citation numbers being automatically sorted and properly
% "compressed/ranged". e.g., [1], [9], [2], [7], [5], [6] without using
% cite.sty will become [1], [2], [5]--[7], [9] using cite.sty. cite.sty's
% \cite will automatically add leading space, if needed. Use cite.sty's
% noadjust option (cite.sty V3.8 and later) if you want to turn this off.
% cite.sty is already installed on most LaTeX systems. Be sure and use
% version 4.0 (2003-05-27) and later if using hyperref.sty. cite.sty does
% not currently provide for hyperlinked citations.
% The latest version can be obtained at:
% http://www.ctan.org/tex-archive/macros/latex/contrib/cite/
% The documentation is contained in the cite.sty file itself.






% *** GRAPHICS RELATED PACKAGES ***
%
\ifCLASSINFOpdf
  % \usepackage[pdftex]{graphicx}
  % declare the path(s) where your graphic files are
  % \graphicspath{{../pdf/}{../jpeg/}}
  % and their extensions so you won't have to specify these with
  % every instance of \includegraphics
  % \DeclareGraphicsExtensions{.pdf,.jpeg,.png}
\else
  % or other class option (dvipsone, dvipdf, if not using dvips). graphicx
  % will default to the driver specified in the system graphics.cfg if no
  % driver is specified.
  % \usepackage[dvips]{graphicx}
  % declare the path(s) where your graphic files are
  % \graphicspath{{../eps/}}
  % and their extensions so you won't have to specify these with
  % every instance of \includegraphics
  % \DeclareGraphicsExtensions{.eps}
\fi
% graphicx was written by David Carlisle and Sebastian Rahtz. It is
% required if you want graphics, photos, etc. graphicx.sty is already
% installed on most LaTeX systems. The latest version and documentation can
% be obtained at: 
% http://www.ctan.org/tex-archive/macros/latex/required/graphics/
% Another good source of documentation is "Using Imported Graphics in
% LaTeX2e" by Keith Reckdahl which can be found as epslatex.ps or
% epslatex.pdf at: http://www.ctan.org/tex-archive/info/
%
% latex, and pdflatex in dvi mode, support graphics in encapsulated
% postscript (.eps) format. pdflatex in pdf mode supports graphics
% in .pdf, .jpeg, .png and .mps (metapost) formats. Users should ensure
% that all non-photo figures use a vector format (.eps, .pdf, .mps) and
% not a bitmapped formats (.jpeg, .png). IEEE frowns on bitmapped formats
% which can result in "jaggedy"/blurry rendering of lines and letters as
% well as large increases in file sizes.
%
% You can find documentation about the pdfTeX application at:
% http://www.tug.org/applications/pdftex





% *** MATH PACKAGES ***
%

% A popular package from the American Mathematical Society that provides
% many useful and powerful commands for dealing with mathematics. If using
% it, be sure to load this package with the cmex10 option to ensure that
% only type 1 fonts will utilized at all point sizes. Without this option,
% it is possible that some math symbols, particularly those within
% footnotes, will be rendered in bitmap form which will result in a
% document that can not be IEEE Xplore compliant!
%
% Also, note that the amsmath package sets \interdisplaylinepenalty to 10000
% thus preventing page breaks from occurring within multiline equations. Use:

% after loading amsmath to restore such page breaks as IEEEtran.cls normally
% does. amsmath.sty is already installed on most LaTeX systems. The latest
% version and documentation can be obtained at:
% http://www.ctan.org/tex-archive/macros/latex/required/amslatex/math/





% *** SPECIALIZED LIST PACKAGES ***
%
\usepackage{algorithmic}
% algorithmic.sty was written by Peter Williams and Rogerio Brito.
% This package provides an algorithmic environment fo describing algorithms.
% You can use the algorithmic environment in-text or within a figure
% environment to provide for a floating algorithm. Do NOT use the algorithm
% floating environment provided by algorithm.sty (by the same authors) or
% algorithm2e.sty (by Christophe Fiorio) as IEEE does not use dedicated
% algorithm float types and packages that provide these will not provide
% correct IEEE style captions. The latest version and documentation of
% algorithmic.sty can be obtained at:
% http://www.ctan.org/tex-archive/macros/latex/contrib/algorithms/
% There is also a support site at:
% http://algorithms.berlios.de/index.html
% Also of interest may be the (relatively newer and more customizable)
% algorithmicx.sty package by Szasz Janos:
% http://www.ctan.org/tex-archive/macros/latex/contrib/algorithmicx/




% *** ALIGNMENT PACKAGES ***
%
%\usepackage{array}
% Frank Mittelbach's and David Carlisle's array.sty patches and improves
% the standard LaTeX2e array and tabular environments to provide better
% appearance and additional user controls. As the default LaTeX2e table
% generation code is lacking to the point of almost being broken with
% respect to the quality of the end results, all users are strongly
% advised to use an enhanced (at the very least that provided by array.sty)
% set of table tools. array.sty is already installed on most systems. The
% latest version and documentation can be obtained at:
% http://www.ctan.org/tex-archive/macros/latex/required/tools/


%\usepackage{mdwmath}
%\usepackage{mdwtab}
% Also highly recommended is Mark Wooding's extremely powerful MDW tools,
% especially mdwmath.sty and mdwtab.sty which are used to format equations
% and tables, respectively. The MDWtools set is already installed on most
% LaTeX systems. The lastest version and documentation is available at:
% http://www.ctan.org/tex-archive/macros/latex/contrib/mdwtools/


% IEEEtran contains the IEEEeqnarray family of commands that can be used to
% generate multiline equations as well as matrices, tables, etc., of high
% quality.


%\usepackage{eqparbox}
% Also of notable interest is Scott Pakin's eqparbox package for creating
% (automatically sized) equal width boxes - aka "natural width parboxes".
% Available at:
% http://www.ctan.org/tex-archive/macros/latex/contrib/eqparbox/





% *** SUBFIGURE PACKAGES ***
%\usepackage[tight,footnotesize]{subfigure}
% subfigure.sty was written by Steven Douglas Cochran. This package makes it
% easy to put subfigures in your figures. e.g., "Figure 1a and 1b". For IEEE
% work, it is a good idea to load it with the tight package option to reduce
% the amount of white space around the subfigures. subfigure.sty is already
% installed on most LaTeX systems. The latest version and documentation can
% be obtained at:
% http://www.ctan.org/tex-archive/obsolete/macros/latex/contrib/subfigure/
% subfigure.sty has been superceeded by subfig.sty.



%\usepackage[caption=false]{caption}
%\usepackage[font=footnotesize]{subfig}
% subfig.sty, also written by Steven Douglas Cochran, is the modern
% replacement for subfigure.sty. However, subfig.sty requires and
% automatically loads Axel Sommerfeldt's caption.sty which will override
% IEEEtran.cls handling of captions and this will result in nonIEEE style
% figure/table captions. To prevent this problem, be sure and preload
% caption.sty with its "caption=false" package option. This is will preserve
% IEEEtran.cls handing of captions. Version 1.3 (2005/06/28) and later 
% (recommended due to many improvements over 1.2) of subfig.sty supports
% the caption=false option directly:
%\usepackage[caption=false,font=footnotesize]{subfig}
%
% The latest version and documentation can be obtained at:
% http://www.ctan.org/tex-archive/macros/latex/contrib/subfig/
% The latest version and documentation of caption.sty can be obtained at:
% http://www.ctan.org/tex-archive/macros/latex/contrib/caption/




% *** FLOAT PACKAGES ***
%
%\usepackage{fixltx2e}
% fixltx2e, the successor to the earlier fix2col.sty, was written by
% Frank Mittelbach and David Carlisle. This package corrects a few problems
% in the LaTeX2e kernel, the most notable of which is that in current
% LaTeX2e releases, the ordering of single and double column floats is not
% guaranteed to be preserved. Thus, an unpatched LaTeX2e can allow a
% single column figure to be placed prior to an earlier double column
% figure. The latest version and documentation can be found at:
% http://www.ctan.org/tex-archive/macros/latex/base/



%\usepackage{stfloats}
% stfloats.sty was written by Sigitas Tolusis. This package gives LaTeX2e
% the ability to do double column floats at the bottom of the page as well
% as the top. (e.g., "\begin{figure*}[!b]" is not normally possible in
% LaTeX2e). It also provides a command:
%\fnbelowfloat
% to enable the placement of footnotes below bottom floats (the standard
% LaTeX2e kernel puts them above bottom floats). This is an invasive package
% which rewrites many portions of the LaTeX2e float routines. It may not work
% with other packages that modify the LaTeX2e float routines. The latest
% version and documentation can be obtained at:
% http://www.ctan.org/tex-archive/macros/latex/contrib/sttools/
% Documentation is contained in the stfloats.sty comments as well as in the
% presfull.pdf file. Do not use the stfloats baselinefloat ability as IEEE
% does not allow \baselineskip to stretch. Authors submitting work to the
% IEEE should note that IEEE rarely uses double column equations and
% that authors should try to avoid such use. Do not be tempted to use the
% cuted.sty or midfloat.sty packages (also by Sigitas Tolusis) as IEEE does
% not format its papers in such ways.


% correct bad hyphenation here
\hyphenation{op-tical net-works semi-conduc-tor}


\usepackage{url}
\usepackage[cmex10]{amsmath}
\interdisplaylinepenalty=2500
\usepackage{amssymb}
\newcommand\Tau{\mathcal{T}}

\begin{document}
%
% paper title
% can use linebreaks \\ within to get better formatting as desired
\title{Phenotyping via Bayesian Nonparametric Tensor Factorization}

%\author{\IEEEauthorblockN{Michael Shell\IEEEauthorrefmark{1},
%Homer Simpson\IEEEauthorrefmark{2},
%James Kirk\IEEEauthorrefmark{3}, 
%Montgomery Scott\IEEEauthorrefmark{3} and
%Eldon Tyrell\IEEEauthorrefmark{4}}
%\IEEEauthorblockA{\IEEEauthorrefmark{1}School of Electrical and Computer Engineering\\
%Georgia Institute of Technology,
%Atlanta, Georgia 30332--0250\\ Email: see http://www.michaelshell.org/contact.html}
%\IEEEauthorblockA{\IEEEauthorrefmark{2}Twentieth Century Fox, Springfield, USA\\
%Email: homer@thesimpsons.com}
%\IEEEauthorblockA{\IEEEauthorrefmark{3}Starfleet Academy, San Francisco, California 96678-2391\\
%Telephone: (800) 555--1212, Fax: (888) 555--1212}
%\IEEEauthorblockA{\IEEEauthorrefmark{4}Tyrell Inc., 123 Replicant Street, Los Angeles, California 90210--4321}}




% use for special paper notices
%\IEEEspecialpapernotice{(Invited Paper)}




% make the title area
\maketitle


\begin{abstract}
%\boldmath
We do phenotyping. 

\end{abstract}
% IEEEtran.cls defaults to using nonbold math in the Abstract.
% This preserves the distinction between vectors and scalars. However,
% if the conference you are submitting to favors bold math in the abstract,
% then you can use LaTeX's standard command \boldmath at the very start
% of the abstract to achieve this. Many IEEE journals/conferences frown on
% math in the abstract anyway.

% no keywords

% For peer review papers, you can put extra information on the cover
% page as needed:
% \ifCLASSOPTIONpeerreview
% \begin{center} \bfseries EDICS Category: 3-BBND \end{center}
% \fi
%
% For peerreview papers, this IEEEtran command inserts a page break and
% creates the second title. It will be ignored for other modes.
\IEEEpeerreviewmaketitle


\section{Introduction}

\section{Model}

\subsection{CANDECOMP/PARAFAC (CP) Tensor Decomposition}

A tensor $T \in R^{n_1 \times n_2 \times \ldots \times n_d}$ can be decomposed by a linear combination of rank-$1$ tensors:
\[
	T = \sum_{i=1}^r \lambda_i a_i^1 \otimes a_i^2 \ldots a_i^d
\]
Where $\lambda_i \in R, a_i^k \in R^{n_k}$ and $\otimes$ represents the outer product. The $r$ is the rank of tensor when $r$ is minimized in above equation. When $r$ is nor minimal, then the above decomposition is referred to as CANDECOMP/PARAFAC decomposition. 

\subsection{Dirichlet Process Tensor Factorization}

\begin{table}[h]
%% increase table row spacing, adjust to taste
%\renewcommand{\arraystretch}{1.3}
\caption{Data Parameters}
\label{data_param}
\centering
\begin{tabular}{|c||c|}
\hline
Notation & Meaning\\
\hline
\textbf{$\lambda$} & The weight vector of phenotypes, here $\|\lambda\| = 1$\\
\hline
$\lambda_k$ & The weight of $k$-th phenotype; $\lambda_k \in [0, 1]$ \\
\hline
$N_p$ & Number of patients \\
\hline
$N_m$ & Number of medications \\
\hline
$N_d$ & Number of diagnosis \\
\hline 
$T$ & Record Tensor. $T \in \{0, 1\}^{N_p \times N_m \times N_d}$ \\
$N$ & Number of Non-zero elements in tensor $T$ \\
\hline
\end{tabular}
\end{table}

\begin{table}[h]
%% increase table row spacing, adjust to taste
%\renewcommand{\arraystretch}{1.3}
\caption{Model Parameters}
\label{model_param}
\centering
\begin{tabular}{|c||c|}
\hline
Notation & Meaning\\
\hline
$\alpha$ & The hyper parameter of Beta Distribution \\
\hline
$\beta_i$ & The stick-breaking process random variable. $\beta_i \sim Beta(1, \alpha)$ \\
\hline
$\gamma^{(p)}$ & The hyper parameter of prior Dirichlet Distribution of patient \\
\hline
$\gamma^{(m)}$ & The hyper parameter of prior Dirichlet Distribution of medication \\
\hline
$\gamma^{(d)}$ & The hyper parameter of prior Dirichlet Distribution of diagnosis \\
\hline
$\theta^{(p, k)}$ & The multinomial distribution over patients given the phenotype $k$ \\
\hline
$\theta^{(m, k)}$ & The multinomial distribution over medications given the phenotype $k$ \\
\hline
$\theta^{(d, k)}$ & The multinomial distribution over diagnosis given the phenotype $k$ \\
\hline
\end{tabular}
\end{table}

We place a stick-breaking process prior on the \textbf{$\lambda$}. The parametere $\beta_i$ follows the Beta Distribution $\beta_i \sim Beta(1, \alpha), i = 1, 2, \ldots$. And:
\[
	\lambda_k = \beta_k \prod_{i=1}^{k-1} (1 - \beta_i), k = 1, 2, \ldots
\]
A $k-$dimensional Dirichlet random variable $\theta$ can take values in the $(k-1)$-simplex where $\sum_{i=1}^k \theta_i = 1$, and the probability density is 
\[
	p(\theta | \gamma) = \frac{\Tau (\sum_{i=1}^k \gamma_i)}{\prod_{i=1}^k \Tau(\gamma_i)} \theta_1^{\gamma_1 - 1}\ldots \theta_k^{\gamma_k - 1}
\]
It is natural to place the conjugate prior (Dirichlet Distribution) over the multinomial distributions parameterized by $\theta^{(p)}, \theta^{(m)}$ and $\theta^{(d)}$:
\begin{align*}
	\theta^{(p, k)} \sim & Dir(\gamma^{(p)}), k = 1, 2, \ldots \\
	\theta^{(m, k)} \sim & Dir(\gamma^{(m)}), k = 1, 2, \ldots \\
	\theta^{(d, k)} \sim & Dir(\gamma^{(d)}), k = 1, 2, \ldots \\
\end{align*}
The patients, medications and diagnosis all follow the multinomial distribution parameterised by $\theta$, specifically:
\begin{align*}
	\text{patient}_i \sim & Multi(\theta^{(p)}), i = 1, 2, \ldots, N_p \\
	\text{medication}_i \sim & Multi(\theta^{(m)}), i = 1, 2, \ldots, N_m \\
	\text{diagnosis}_i \sim & Multi(\theta^{(d)}), i = 1, 2, \ldots, N_d \\
\end{align*}

Given the parameters, the likelihood of given record triplet $(i, j, k)$ (patient $i$ had diagnostic record $k$, and has taken medicine $j$) can be formulated as:
	\begin{align*}
& p(T_{i, j, k} | \alpha, \gamma^{(p)}, \gamma^{(m)}, \gamma^{(d)}) \\
=& \int p(\beta | \alpha) \sum_{r=1}^{\infty} \lambda_r 
p(i | r, \gamma^{(p)})p(j | r, \gamma^{(m)})p(k | r, \gamma^{(d)}) d\beta
	\end{align*}

\subsection{Collapsed Gibbs Sampling}

We define a set $S = \{(p_n, m_n, d_n)\}, n = 1, \ldots, N$, where $T(p_n, m_n, d_n) = 1$, consists of triplets representing the coordinates of non-zero elements in tensor $T$. Our model is a type of mixture model, so it is natural to define a hidden variable $z_n, n = 1, \ldots, N$ which representing the mixture index (i.e., the index of phenotype) of record triplet $(p_n, m_n, d_n)$. $z$ is a vector of length $N$ and $z_t$ represents its $t$th element. 

Directly find the MAP of above equation is intractable, so we consider using MCMC to sample the parameters from posterior distribution. Here we use Gibbs sampling technique, then we need the posterior probability of a certain hidden variable $z_t$, $p(z_t | z_{-t}, S, \alpha, \gamma^{(p)}, \gamma^{(m)}, \gamma^{(d)})$. Here $z_{-t}$ means the hidden variable set without $t$th element.  

Let's consider a simpler case, when the number if phenotypes is $R$, a finite number. Then $\lambda \sim Dir(\frac{\alpha}{R})$. 
Use definition of conditional probability, 
\begin{align*}
	p(z_t | z_{-t}, S, \alpha, \gamma^{(p)}, \gamma^{(m)}, \gamma^{(d)}) = & 
	\frac{p(z_t, z_{-t}, S | \alpha, \gamma^{(p)}, \gamma^{(m)}, \gamma^{(d)})}{p(z_{-t}, S | \alpha, \gamma^{(p)}, \gamma^{(m)}, \gamma^{(d)})} \\
	\propto & p(z_t , z_{-t}, S | \alpha, \gamma^{(p)}, \gamma^{(m)}, \gamma^{(d)}) \\
	= & p(z, S | \alpha, \gamma^{(p)}, \gamma^{(m)}, \gamma^{(d)})
\end{align*}
Expand it using rule of total probability, 
\begin{align*}
 = & \int \int \int \int p(z, S, \lambda | \theta^{(p)}, \theta^{(m)}, \theta^{(d)}) d\lambda d\theta^{(p)} d\theta^{(m)} d\theta^{(d)} \\
 = & \int \int \int \int p(z | \lambda) p(\lambda | \alpha) p(S | \theta^{(p)}, \theta^{(m)}, \theta^{(d)}, z) \\
  & p(\theta^{(p)} | \gamma^{(p)}) p(\theta^{(m)} | \gamma^{(m)}) p(\theta^{(d)} | \gamma^{(d)}) d\lambda d\theta^{(p)} d\theta^{(m)} d\theta^{(d)} \\
 = & \int p(z | \lambda) p(\lambda | \alpha) d\lambda \times 
  \int \int \int \prod_{n=1}^N p(S_n | \theta^{(p)}, \theta^{(m)}, \theta^{(d)}, z) \\
 & p(\theta^{(p)} | \gamma^{(p)}) p(\theta^{(m)} | \gamma^{(m)}) p(\theta^{(d)} | \gamma^{(d)}) d\lambda d\theta^{(p)} d\theta^{(m)} d\theta^{(d)} \\
\end{align*}
We use a indicator function $I(\cdot): Bool \rightarrow \{0, 1\}$. It returns 1 iff the statement is true. Then we can further expand the above term:
\begin{align*}
= & \int p(z | \lambda) p(\lambda | \alpha) d\lambda \times \prod_{r=1}^R \int \int \int \\
 & \prod_{n=1}^N p(S_n | \theta^{(p, r)}, \theta^{(m, r)}, \theta^{(d, r)})^{I(z_n = r)} \\
 & p(\theta^{(p)} | \gamma^{(p)}) p(\theta^{(m)} | \gamma^{(m)}) p(\theta^{(d)} | \gamma^{(d)}) d\lambda d\theta^{(p)} d\theta^{(m)} d\theta^{(d)} \\
\end{align*} 
Define a count operation $C_{r, *, *, *}$ which represents the number of all the records in $S$ whose phenotype index is $r$. We can further define $C_{r, i, *, *}$ to represents the number of all the records in $S$ whose phenotype index is $r$ and the patient id is $i$. Similarly we have $C_{r, *, j, *}$ and $C_{r, *, *, k}$. Then:
\begin{align*}
= & \int p(\lambda | \alpha) \prod_{r=1}^R \lambda_r^{C_{r, *, *, *}} d\lambda \\ 
& \times \prod_{r=1}^R \int p(\theta^{(p, r)} | \gamma^{(p)}) \prod_{i=1}^{N_p} (\theta_i^{(p, r)})^{C_{r, i, *, *}} d\theta^{(p, r)} \\
& \times \prod_{r=1}^R \int p(\theta^{(m, r)} | \gamma^{(m)}) \prod_{j=1}^{N_m} (\theta_i^{(m, r)})^{C_{r, *, j, *}} d\theta^{(m, r)} \\
& \times \prod_{r=1}^R \int p(\theta^{(d, r)} | \gamma^{(d)}) \prod_{k=1}^{N_d} (\theta_i^{(d, r)})^{C_{r, *, *, k}} d\theta^{(d, r)} 
\end{align*}
Plugin the Dirichlet prior, the first term can be further simplified as:
\begin{align*}
& \int \frac{\Tau(\alpha)}{\prod_{r=1}^R\Tau(\frac{\alpha}{R})} \prod_{r=1}^R \lambda_r^{\frac{\alpha}{R} - 1} \prod_{r=1}^R \lambda_r^{C_{r, *, *, *}}
 d\lambda  \\
= & \frac{\Tau(\alpha)}{\prod_{r=1}^R\Tau(\frac{\alpha}{R})} \int \prod_{r=1}^R \lambda_r^{C_{r, *, *, *} + \frac{\alpha}{R} - 1} d\lambda \\
= & \frac{\Tau(\alpha)}{\prod_{r=1}^R\Tau(\frac{\alpha}{R})} \frac{\prod_{r=1}^R\Tau(C_{r, *, *, *} + \frac{\alpha}{R})}{\Tau(N + \alpha)} \\
 & \int \frac{\Tau(N + \alpha)}{\prod_{r=1}^R\Tau(C_{r, *, *, *} + \frac{\alpha}{R})} \prod_{r=1}^R \lambda_r^{C_{r, *, *, *} + \frac{\alpha}{R} - 1} d\lambda \\
= & \frac{\Tau(\alpha)}{\prod_{r=1}^R\Tau(\frac{\alpha}{R})} \frac{\prod_{r=1}^R\Tau(C_{r, *, *, *} + \frac{\alpha}{R})}{\Tau(N + \alpha)} \\
\propto & \frac{\prod_{r=1}^R\Tau(C_{r, *, *, *} + \frac{\alpha}{R})}{\Tau(N + \alpha)} \\
= & \frac{\Tau(C_{z_t, *, *, *}^{-t} + \frac{\alpha}{R} + 1) \times \prod_{r \neq z_t} \Tau (C_{r, *, *, *} + \frac{\alpha}{R})}{\Tau(N + \alpha)} \\
= & (C_{z_t, *, *, *}^{-t} + \frac{\alpha}{R})\frac{\Tau(C_{z_t, *, *, *}^{-n} + \frac{\alpha}{R}) \times \prod_{r \neq z_t} \Tau (C_{r, *, *, *} + \frac{\alpha}{R})}{\Tau(N + \alpha)} \\
\propto & C_{z_t, *, *, *}^{-t} + \frac{\alpha}{R}
\end{align*}
We use $\Tau(x + 1) = x \Tau(x)$ here. $C_{z_t, *, *, *}^{-t}$ means the counting result without the $t$-th record. So $C_{z_t, *, *, *}^{-t} + 1$ equals to $C_{z_t, *, *, *}$. Similarly, we can simplify the second, third and fourth term, using the above tricks:
\begin{align*}
&	\prod_{r=1}^R \int p(\theta^{(p, r)} | \gamma^{(p)}) \prod_{i=1}^{N_p} (\theta_i^{(p, r)})^{C_{r, i, *, *}} d\theta^{(p, r)} \\
\propto & \frac{C_{z_t, p_t, *, *}^{-t} + \gamma_{p_t}^{(p)}}{\sum_{i=1}^{N_p}C_{z_t, i, *, *}^{-t} + \gamma_{i}^{(p)}} \text{ ,}\\
&	\prod_{r=1}^R \int p(\theta^{(m, r)} | \gamma^{(m)}) \prod_{j=1}^{N_m} (\theta_i^{(m, r)})^{C_{r, *, j, *}} d\theta^{(m, r)} \\
\propto & \frac{C_{z_t, *, m_t, *}^{-t} + \gamma_{m_t}^{(m)}}{\sum_{j=1}^{N_m}C_{z_t, *, j, *}^{-t} + \gamma_{j}^{(m)}} \text{ and}\\
&	\prod_{r=1}^R \int p(\theta^{(d, r)} | \gamma^{(d)}) \prod_{k=1}^{N_d} (\theta_i^{(d, r)})^{C_{r, *, *, k}} d\theta^{(d, r)} \\
\propto & \frac{C_{z_t, *, *, d_t}^{-t} + \gamma_{d_t}^{(d)}}{\sum_{k=1}^{N_d}C_{z_t, *, *, k}^{-t} + \gamma_{k}^{(d)}}
\end{align*}
So 
\begin{align*}
&  	p(z_t | z_{-t}, S, \alpha, \gamma^{(p)}, \gamma^{(m)}, \gamma^{(d)}) \\
\propto & (C_{z_t, *, *, *}^{-t} + \frac{\alpha}{R})
\times \frac{C_{z_t, p_t, *, *}^{-t} + \gamma_{p_t}^{(p)}}{\sum_{i=1}^{N_p}C_{z_t, i, *, *}^{-t} + \gamma_{i}^{(p)}} \\
& \times \frac{C_{z_t, *, m_t, *}^{-t} + \gamma_{m_t}^{(m)}}{\sum_{j=1}^{N_m}C_{z_t, *, j, *}^{-t} + \gamma_{j}^{(m)}} 
\times \frac{C_{z_t, *, *, d_t}^{-t} + \gamma_{d_t}^{(d)}}{\sum_{k=1}^{N_d}C_{z_t, *, *, k}^{-t} + \gamma_{k}^{(d)}}
\end{align*}

Since $\sum_{r=1}^{\infty} C_{z_t, *, *, *}^{-t} + \frac{\alpha}{R} = N + \alpha - 1$, when $R \rightarrow \infty$, the weight of existing mixture is $\frac{C_{z_t, *, *, *}^{-t}}{N + \alpha - 1}$, and the weight of new mixture is $\frac{\alpha}{N + \alpha - 1}$. 

\section{Algorithm}

\subsection{Single thread Gibbs sampling}

Now since we have the marginal posterior probability, we can run Gibbs sampling to inference the hidden variables $z_n, n = 1, \ldots, N$. Define $f(t) = \frac{C_{z_t, p_t, *, *}^{-t} + \gamma_{p_t}^{(p)}}{\sum_{i=1}^{N_p}C_{z_t, i, *, *}^{-t} + \gamma_{i}^{(p)}} 
 \times \frac{C_{z_t, *, m_t, *}^{-t} + \gamma_{m_t}^{(m)}}{\sum_{j=1}^{N_m}C_{z_t, *, j, *}^{-t} + \gamma_{j}^{(m)}} 
\times \frac{C_{z_t, *, *, d_t}^{-t} + \gamma_{d_t}^{(d)}}{\sum_{k=1}^{N_d}C_{z_t, *, *, k}^{-t} + \gamma_{k}^{(d)}}$. The algorithm is shown below: 
	
	\begin{itemize}
		\item[1.] Set $R = R_0$, where $R_0$ is the initial number of phenotypes; Randomly initialize $z$. 
		\item[2.] Count the values $C_{(\cdot, \cdot, \cdot, \cdot)}$. 
		\item[3.] For $t \in 1, 2, \ldots, N$
			\begin{itemize}
				\item[a)] Remove $S_t = (p_t, m_t, d_t)$ from data. Update $C_{(\cdot, \cdot, \cdot, \cdot)}$
				\item[b)] If $S_t$ is the only instance in mixture $z_t$, then remove this mixture, and reduce $R$ by one. 
				\item[c)] Draw $z_t$: 
				\begin{align*}
					p(z_t = r \leq R) \propto &  \frac{C_{z_t, *, *, *}^{-t}}{N + \alpha - 1} \times f(t) \\
					p(z_t = R + 1) \propto & \frac{\alpha}{N + \alpha - 1} \times f(t)
				\end{align*}
			\end{itemize}
		\item[4.] Update $R$ if $z_t = R + 1$. 
		\item[5.] Update $C_{(\cdot, \cdot, \cdot, \cdot)}$ accordingly. 
	\end{itemize}
 



% An example of a floating figure using the graphicx package.
% Note that \label must occur AFTER (or within) \caption.
% For figures, \caption should occur after the \includegraphics.
% Note that IEEEtran v1.7 and later has special internal code that
% is designed to preserve the operation of \label within \caption
% even when the captionsoff option is in effect. However, because
% of issues like this, it may be the safest practice to put all your
% \label just after \caption rather than within \caption{}.
%
% Reminder: the "draftcls" or "draftclsnofoot", not "draft", class
% option should be used if it is desired that the figures are to be
% displayed while in draft mode.
%
%\begin{figure}[!t]
%\centering
%\includegraphics[width=2.5in]{myfigure}
% where an .eps filename suffix will be assumed under latex, 
% and a .pdf suffix will be assumed for pdflatex; or what has been declared
% via \DeclareGraphicsExtensions.
%\caption{Simulation Results}
%\label{fig_sim}
%\end{figure}

% Note that IEEE typically puts floats only at the top, even when this
% results in a large percentage of a column being occupied by floats.


% An example of a double column floating figure using two subfigures.
% (The subfig.sty package must be loaded for this to work.)
% The subfigure \label commands are set within each subfloat command, the
% \label for the overall figure must come after \caption.
% \hfil must be used as a separator to get equal spacing.
% The subfigure.sty package works much the same way, except \subfigure is
% used instead of \subfloat.
%
%\begin{figure*}[!t]
%\centerline{\subfloat[Case I]\includegraphics[width=2.5in]{subfigcase1}%
%\label{fig_first_case}}
%\hfil
%\subfloat[Case II]{\includegraphics[width=2.5in]{subfigcase2}%
%\label{fig_second_case}}}
%\caption{Simulation results}
%\label{fig_sim}
%\end{figure*}
%
% Note that often IEEE papers with subfigures do not employ subfigure
% captions (using the optional argument to \subfloat), but instead will
% reference/describe all of them (a), (b), etc., within the main caption.


% An example of a floating table. Note that, for IEEE style tables, the 
% \caption command should come BEFORE the table. Table text will default to
% \footnotesize as IEEE normally uses this smaller font for tables.
% The \label must come after \caption as always.
%
%\begin{table}[!t]
%% increase table row spacing, adjust to taste
%\renewcommand{\arraystretch}{1.3}
% if using array.sty, it might be a good idea to tweak the value of
% \extrarowheight as needed to properly center the text within the cells
%\caption{An Example of a Table}
%\label{table_example}
%\centering
%% Some packages, such as MDW tools, offer better commands for making tables
%% than the plain LaTeX2e tabular which is used here.
%\begin{tabular}{|c||c|}
%\hline
%One & Two\\
%\hline
%Three & Four\\
%\hline
%\end{tabular}
%\end{table}


% Note that IEEE does not put floats in the very first column - or typically
% anywhere on the first page for that matter. Also, in-text middle ("here")
% positioning is not used. Most IEEE journals/conferences use top floats
% exclusively. Note that, LaTeX2e, unlike IEEE journals/conferences, places
% footnotes above bottom floats. This can be corrected via the \fnbelowfloat
% command of the stfloats package.

\section{Experiments}


\section{Conclusion}

\section*{Acknowledgment}


% The IEEEtran BibTeX style support page is at:
% http://www.michaelshell.org/tex/ieeetran/bibtex/
%\bibliographystyle{IEEEtran}
% argument is your BibTeX string definitions and bibliography database(s)
%\bibliography{IEEEabrv,../bib/paper}
%
% <OR> manually copy in the resultant .bbl file
% set second argument of \begin to the number of references
% (used to reserve space for the reference number labels box)
\begin{thebibliography}{1}

\bibitem{IEEEhowto:kopka}
H.~Kopka and P.~W. Daly, \emph{A Guide to \LaTeX}, 3rd~ed.\hskip 1em plus
  0.5em minus 0.4em\relax Harlow, England: Addison-Wesley, 1999.

\end{thebibliography}


\end{document}


